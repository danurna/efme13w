\documentclass{article}
%\documentclass[journal]{IEEEtran}
%\documentclass{report}
%\documentclass{acta}

\usepackage{graphicx}
\usepackage[utf8]{inputenc}
\usepackage[T1]{fontenc}

\begin{document}

\title{Zusammenfassung \\ \textit{When Is "Nearest Neighbor" Meaningful? \\ Beyer, Goldstein, Ramakrishnan, Shaft.}}
\author{Gruppe B9}

\maketitle

\begin{abstract}
Diese Zusammenfassung ist im Zuge der Übung zur Einführung in die Mustererkennung WS 2013 entstanden. Das zusammenzufassende Werk trägt den Titel "When Is 'Nearest Neighbor' Meaningful?" von Beyer, Goldstein, Ramakrishnan, Shaft und wurde an der University of Wisconsin-Madison veröffentlicht.
\end{abstract}

\vfill

Gruppe B9

Florian Groh, 1168186

Felix Ledochowski, 1028318

Daniel Witurna, 1125818

\pagebreak

\section{Zusammenfassung des Papers}

Der Artikel behandelt das Problem von hohen Dimensionen bei der Verwendung eines Nearest-Neighbor Algorithmus. Es werden Datensätze vorgestellt, die empirisch zeigen, dass schon ab 10-15 Dimension eine Anwendung von NN-Verfahren kein wertvolles Ergebnis liefern kann. Dies wird insbesondere zum Problem, da immer mehr versucht wird, schwierige und komplexe Daten durch hoch-dimensionale Merkmalsvektoren anzunähern.
\\
\\
Um die Wertigkeit von Datensätzen und damit auch deren Eignung für NN zu bestimmen, führen Beyer et al. die Definition der Instabilität ein. Instabiles Verhalten entsteht, sobald ein Großteil der Nachbarn eines Anfragepunktes kaum weiter entfernt sind als der nächste Nachbar (Nearest Neighbor). Um dies ein wenig formeller auszudrücken führen wir die Begriffe Minimal- und Maximaldistanz ein. Bei der Minimaldistanz (Dmin) handelt es sich um die Entfernung zwischen dem Anfragepunkt und dem nächsten Nachbarn, die Maximaldistanz (Dmax) ist dementsprechend der Nachbar mit der weitesten Entfernung. Ein Datensatz ist genau dann instabil, wenn für einen beliebigen Anfragepunkt die Maximaldistanz um einen kleinen Faktor Epsilon > 0 größer als die Minimaldistanz ist (Dmax = (1+Epsilon) * Dmin).
\\
Von stabilem Verhalten wird dann gesprochen, wenn wenige Punkte in diese erweiterte Umgebung fallen und dadurch die NN-Anfrageumgebung von den anderen Daten sinnvoll getrennt ist. Diese Definition wird verwendet, um zu zeigen, dass in vielen Situationen, in denen die Dimensionalität ansteigt, die Wahrscheinlichkeit der Instabilität einer NN-Anfrage gegen 1 konvergiert. Diese eben erwähnten Situationen treten nur unter Zutreffen der von Bayer et al. aufgestellten Theoreme auf.
\\
\\
Auch auf die Frage welche Datensätze auch bei hoch-dimensionaler Indexierung für den NN-Algorithmus Sinn ergeben gehen die Autoren ein. Beispielsweise macht eine Klassifizierung sehr wohl Sinn, falls Trainings-Werte existieren, die genau der Anfrage entsprechen und dadurch die minimale Distanz 0 wird. Eine andere Möglichkeit bietet sich, wenn Punkte der Anfrage nicht genau übereinstimmen müssen sondern auch in einer kleinen Entfernung eines Datenpunkts erlaubt sind. Je mehr Dimensionen die Datenpunkte allerdings haben, umso wahrscheinlicher wird, dass selbst sorgfältig ausgewählte Trainings-Sets nicht mehr diskriminative Entfernungen haben. Weitere Ausführungen beschäftigen sich mit der Bildung von sogenannten Clustern, in welche die Anfrage fallen muss.
\\
\\
Als Ergebnis dieses Papers werden mehrere Punkte angeführt. Beispielsweise wurde mehrere Szenarien der Anwendung gefunden, in denen die Entfernungen von Punkten und deren nächsten Nachbarn vernachlässigbar klein werden. In diese Anwendungsgebiete fällt auch die übliche und vielgeprüfte Verwendung von NN als Heuristik.
Weiters war die Frage in welchen Bereichen die Dimensionalität so hoch ist, dass kein aussagekräftiges Ergebnis mehr gefunden wird. Die Antwort konnten die Autoren durch Simulationen bestimmen und meinen, dass in den ersten 20 Dimensionen der Abstand der Punkte sehr stark abnimmt und über die 20. Dimension sehr schnell einen Punkt erreicht, ab dem die Entfernung zu gering wird.
Der praktische Nutzen dieser Arbeit wird in den Bereichen der Evaluierung eines Nearest Neighbor - Aufgabenbereich sowie einer Nearest Neighbor - Vorgehensweise angegeben.


\end{document}
